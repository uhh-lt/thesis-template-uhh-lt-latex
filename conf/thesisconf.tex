% !TEX root = ../main.tex
% !TEX encoding = UTF-8 Unicode
% !TEX TS-program = pdflatexmk
% !TEX spellcheck = en-US
% !BIB program = biber

\usepackage[ngerman,main=english]{babel}

\usepackage[utf8]{inputenc}
% !TEX encoding = UTF-8 Unicode
% !TEX root = ../diss.tex
% !TEX spellcheck = en-US
% !BIB program = biber

% save conflicting degree command from ociamthesis
\let\thesisdegree=\degree

\usepackage[sb]{libertine} % or \usepackage[sb]{libertinus}
\usepackage[T1]{fontenc}
\usepackage{textcomp}
\usepackage[varqu,varl]{zi4}% inconsolata for mono, not LibertineMono
\usepackage[amsthm]{libertinust1math} % slanted integrals, by default
\usepackage[scr=boondoxo,bb=boondox]{mathalpha} %Omit bb=boondox for default libertinus bb

% make degree command from ociamthesis the main degree command, save degree in libertinust1math as \mathdegree
\let\mathdegree=\degree
\let\degree=\thesisdegree

% Keine "Schusterjungen" (einzelne Zeilen am Ende einer Seite)
\clubpenalty=10000
% Keine "Hurenkinder" (einzelne Zeilen am Anfang einer Seite)
\widowpenalty=10000
\displaywidowpenalty=10000



%%%%% SELECT YOUR DRAFT OPTIONS
%\degreedate{April, 2023}
%\degreedate{~}
\degreedate{\emph{DRAFT Printed on \today}}

% Footer
\fancyfoot[C]{\emph{DRAFT Printed on \today}}

% use corrections as delta highlighting mechanism (see ociamthesis)
% This highlights (in blue) corrections marked with (for words) \mccorrect{blah} or (for whole
% paragraphs) \begin{mccorrection} . . . \end{mccorrection}.
%\correctionstrue

% BIBLIOGRAPHY resources
\addbibresource{./bib/biblio-clean.bib}
%\addbibresource{./bib/biblio-temp.bib}

% Uncomment this if you want equation numbers per section (2.3.12), instead of per chapter (2.18):
%\numberwithin{equation}{subsection}

%%%%% THESIS / TITLE PAGE INFORMATION
\newcommand{\mytitle}{Fancy \\ Thesis \\ Title}
\title{\mytitle}

% title without line breaks
\newcommand{\mytitleplain}{Fancy Thesis Title}
\titleplain{\mytitleplain}

% authorname
\newcommand{\authorname}{Allan M. Turing}

\dateofsubmission{1.1.2099}
\dateofdisputation{ 2.1.2099 }
\typeofthesis{Dissertation}
\supervisorname{Prof.\ Dr.\ Chris Biemann, Universität Hamburg}
\comittee{%
  \begin{tabular}{@{}l@{~~}l@{~}l@{~}l@{}}%
    & 1\textsuperscript{st} & Reviewer: & Prof.\ Dr.\ Chris Biemann, Universität Hamburg \\%
    & 2\textsuperscript{nd} & Reviewer: & Prof.\ Dr.\ Konrad Zuse, Universität Hamburg \\%
    & \multicolumn{2}{@{}r@{~}}{Chair:} & Prof.\ Dr.\ Albert Einstein, Universität Hamburg %
\end{tabular}%
}%
\university{Universit\"a{}t Hamburg\\Hamburg, Germany}%
\address{Hamburg, Germany}%
\faculty{Faculty of Mathematics, Informatics and Natural Sciences}
\department{Department of Informatics}
\researchgroup{Language Technology}
\universitylogo{%
\begin{minipage}{\textwidth}%
 \centering%
 \includegraphics[width=.7\textwidth]{figures/up-uhh-logo-u-2010-u-farbe-u-cmyk-modus}%
 \end{minipage}%
}%

% Your full degree name.  (But remember that DPhils aren't "in" anything.  They're just DPhils.)
\degree{Doctor rerum naturalium (Dr.\ rer.\ nat.)}

% set title and author
\author{\authorname}

%%%%% PDF METADATA
\hypersetup{%
  pdftitle={\mytitleplain},%
  pdfauthor={\authorname},%
  pdfsubject={\mytitleplain},%
  pdfview=FitH,%
  pdfstartview=FitV,%
  pdfproducer={\authorname},%
  pdfcreator=\textsc{LaTeX},%
	colorlinks=true,% remove boxes for links
	citecolor=black,%
	linkcolor=LimeGreen,% black % make all colors black for the final submission
	anchorcolor=LimeGreen,% black %
	filecolor=LimeGreen,% black %
	runcolor=LimeGreen,% black %
	urlcolor=LimeGreen,% black %
	menucolor=LimeGreen,% black %
}%

%%%%% MORE MACROS
\usepackage{makeidx}
\usepackage{placeins}
\usepackage[labelfont={footnotesize,up,tt}]{subcaption} % subrefformat=parens,
\usepackage{kantlipsum}
\usepackage{lipsum}
\usepackage{url}
\usepackage{amssymb}
\usepackage{amsmath}
\usepackage{booktabs}
\usepackage{listings}
\usepackage[most]{tcolorbox}
\usepackage{amsthm}
\theoremstyle{definition}
\newtheorem{definition}{Definition}%[section]

%%%%% MORE COMMANDS, REDEFINES, ENVIRONMENTS, ...

% define hangindent for listitems
\newlength\listindent
\setlength\listindent{13pt}
\newcommand{\hangindentlistitems}{\parshape 2 0cm \linewidth \listindent \dimexpr\linewidth-\listindent\relax}

%%%
\graphicspath{%
  {figures/}%
  {figures/ch-1/}%
  {figures/sample/}%
}%


%% manual hyphenation rules
\hyphenation{down-stream down-stream-tasks}
