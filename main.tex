% !TEX root = main.tex
% !TEX encoding = UTF-8 Unicode
% !TEX TS-program = pdflatexmk
% !TEX spellcheck = en-US
% !BIB program = biber

%%%%%%%%%%%%%%%%%%%%%%%%%%%%%%%%%%%%%%%%%%%%%%%%%%%%%%%%%%%%%%%
%% UHH LT THESIS TEMPLATE based on the OXFORD THESIS TEMPLATE

%%%%% CHOOSE PAGE LAYOUT

% two-sided binding:
% \documentclass[a4paper,twoside]{conf/uhhltthesis}
% one-sided binding:
% \documentclass[a4paper]{conf/uhhltthesis}
% PDF output (ie equal margins, no extra blank pages, for online publication):
\documentclass[hidelinks, a4paper, nobind]{conf/uhhltthesis}

% !TEX root = ../main.tex
% !TEX encoding = UTF-8 Unicode
% !TEX TS-program = pdflatexmk
% !TEX spellcheck = en-US
% !BIB program = biber

\usepackage[ngerman,main=english]{babel}
% !TEX encoding = UTF-8 Unicode
% !TEX root = ../diss.tex
% !TEX spellcheck = en-US
% !BIB program = biber

% save conflicting degree command from ociamthesis
\let\thesisdegree=\degree

\usepackage[sb]{libertine} % or \usepackage[sb]{libertinus}
\usepackage[T1]{fontenc}
\usepackage{textcomp}
\usepackage[varqu,varl]{zi4}% inconsolata for mono, not LibertineMono
\usepackage[amsthm]{libertinust1math} % slanted integrals, by default
\usepackage[scr=boondoxo,bb=boondox]{mathalpha} %Omit bb=boondox for default libertinus bb

% make degree command from ociamthesis the main degree command, save degree in libertinust1math as \mathdegree
\let\mathdegree=\degree
\let\degree=\thesisdegree

% Keine "Schusterjungen" (einzelne Zeilen am Ende einer Seite)
\clubpenalty=10000
% Keine "Hurenkinder" (einzelne Zeilen am Anfang einer Seite)
\widowpenalty=10000
\displaywidowpenalty=10000


%%%%% SELECT YOUR DRAFT OPTIONS
%\degreedate{April, 2024}
%\degreedate{~}
\degreedate{\emph{DRAFT Printed on \today}}

% Footer
\fancyfoot[C]{\emph{DRAFT Printed on \today}}

% use corrections as delta highlighting mechanism (see ociamthesis)
% This highlights (in blue) corrections marked with (for words) \mccorrect{blah} or (for whole
% paragraphs) \begin{mccorrection} . . . \end{mccorrection}.
%\correctionstrue

% BIBLIOGRAPHY resources
\addbibresource{./bib/biblio-clean.bib}
%\addbibresource{./bib/biblio-temp.bib}

% Uncomment this if you want equation numbers per section (2.3.12), instead of per chapter (2.18):
%\numberwithin{equation}{subsection}

%%%%% THESIS / TITLE PAGE INFORMATION
\newcommand{\mytitle}{Fancy \\ Thesis \\ Title}
\title{\mytitle}

% title without line breaks
\newcommand{\mytitleplain}{Fancy Thesis Title}
\titleplain{\mytitleplain}

% authorname
\newcommand{\authorname}{Allan M. Turing}

% specifiy Date of submission
\dateofsubmission{ 1.1.2099 }

% specifiy Date of disputation if necessary (usually not necessary for bachelor or master degree)
\dateofdisputation{ 2.1.2099 } 

% who's your supervisor?
\supervisonby{John von Neumann, Universität Hamburg}

% who's part of your comittee?
\comittee{%
  1\textsuperscript{st} Examiner: Prof.\ Dr.\ Chris Biemann, Universität Hamburg \\%
  2\textsuperscript{nd} Examiner: Dr.\ Konrad Zuse, Universität Hamburg \\%
  ...
}%
%% OVERRIDE defaults:
%\university{Universit\"a{}t Hamburg}%
%\unviersitylogo{\includegraphics[width=\textwidth,clip,trim=1.95cm 2.5cm 1.95cm 2.5cm]{figures/up-uhh-logo-u-2010-u-farbe-u-cmyk-modus}}
%\address{Hamburg, Germany}%
\faculty{Faculty of Mathematics, Informatics and Natural Sciences}
%\facultylogo{\includegraphics[width=\textwidth]{figures/uhh-min-faculty-de}}
% \department{Department of Informatics}
% \researchgroup{Language Technology}
%\grouplogo{\includegraphics[width=\textwidth]{figures/uhh-lt-logo}}


% set author
\author{\authorname}

%%%%% PDF METADATA
\hypersetup{%
  pdftitle={\mytitleplain},%
  pdfauthor={\authorname},%
  pdfsubject={\mytitleplain},%
  pdfview=FitH,%
  pdfstartview=FitV,%
  pdfproducer={\authorname},%
  pdfcreator=\textsc{LaTeX},%
	colorlinks=true,% remove boxes for links
	citecolor=black,%
	linkcolor=LimeGreen,% black % NOTE: make all colors black for the final print submission
	anchorcolor=LimeGreen,% black %
	filecolor=LimeGreen,% black %
	runcolor=LimeGreen,% black %
	urlcolor=LimeGreen,% black %
	menucolor=LimeGreen,% black %
}%

%%%%% MORE MACROS
\usepackage{makeidx}
\usepackage{placeins}
\usepackage[labelfont={footnotesize,up,tt}]{subcaption} % subrefformat=parens,
\usepackage{kantlipsum}
\usepackage{lipsum}
\usepackage{url}
\usepackage{amssymb}
\usepackage{amsmath}
\usepackage{booktabs}
\usepackage{listings}
\usepackage{amsthm}
\theoremstyle{definition}
\newtheorem{definition}{Definition}%[section]

%%%%% MORE COMMANDS, REDEFINES, ENVIRONMENTS, ...

% define hangindent for listitems
\newlength\listindent
\setlength\listindent{13pt}
\newcommand{\hangindentlistitems}{\parshape 2 0cm \linewidth \listindent \dimexpr\linewidth-\listindent\relax}

%%%
\graphicspath{%
  {figures/}%
  {figures/ch-1/}%
  {figures/sample/}%
}%

%% manual hyphenation rules
\hyphenation{down-stream down-stream-tasks}


\makeindex

%%%%% THE ACTUAL DOCUMENT STARTS HERE
\begin{document}

%%%%% CHOOSE YOUR LINE SPACING HERE
% Zeilenabstand Einstellung
% This is the official option.  Use it for your submission copy and library copy:
%\setlength{\textbaselineskip}{22pt plus2pt}
% This is closer spacing (about 1.5-spaced) that you might prefer for your personal copies:
%\setlength{\textbaselineskip}{18pt plus2pt minus1pt}
\setlength{\textbaselineskip}{\baselineskip}

% You can set the spacing here for the roman-numbered pages (acknowledgements, table of contents, etc.)
%\setlength{\frontmatterbaselineskip}{17pt plus1pt minus1pt}
\setlength{\frontmatterbaselineskip}{\baselineskip}

% Leave this line alone; it gets things started for the real document.
\setlength{\baselineskip}{\textbaselineskip}


%%%%% CHOOSE YOUR SECTION NUMBERING DEPTH HERE
% 0 = chapter; 1 = section; 2 = subsection; 3 = subsubsection, 4 = paragraph...
% The level that gets a number:
\setcounter{secnumdepth}{2}
% The level that shows up in the ToC:
\setcounter{tocdepth}{2}

%%%%% ABSTRACT SEPARATE - an abstract that is separate from the thesis
%\begin{abstractseparate}
%	% !TEX root = ../main.tex
% !TEX encoding = UTF-8 Unicode
% !TEX TS-program = pdflatexmk
% !TEX spellcheck = en-US
% !BIB program = biber

\kant[1-3]


 % Create an abstract.tex file in the 'text' folder for your abstract.
%\end{abstractseparate}

% Pages are roman numbered from here, though page numbers are invisible until ToC
\begin{romanpages}

% Title page is created here (includes affidavit)
\maketitle

%%%%% DEDICATION -- If you'd like, un-comment the following.
\begin{dedication}
  This work is dedicated to some important person(s) for some important reason
\end{dedication}

%%%%% ACKNOWLEDGEMENTS -- If you'd like, un-comment the following.
\begin{acknowledgements}
  % !TEX root = ../main.tex
% !TEX encoding = UTF-8 Unicode
% !TEX TS-program = pdflatexmk
% !TEX spellcheck = en-US
% !BIB program = biber

\subsection*{Personal}

This is where you thank your advisor, colleagues, and family and friends.

\subsection*{Institutional}

If you want to separate out your thanks for funding and institutional support, I don't think there's any rule against it.  Of course, you could also just remove the subsections and do one big traditional acknowledgement section.

\end{acknowledgements}

%%%%% THEMED QUOTE -- If you'd like one, un-comment the following.
\begin{themedquote}{Johann Wolfgang von Goethe, 1829}
  Alles Gescheite ist schon gedacht worden.\\
Man muss nur versuchen, es noch einmal zu denken. \\[\baselineskip]

All intelligent thoughts have already been thought;\\
what is necessary is only to try to think them again.
\end{themedquote}

%%%%% ABSTRACT -- Nothing to do here except comment out if you don't want it.
\begin{abstract}
  % !TEX root = ../main.tex
% !TEX encoding = UTF-8 Unicode
% !TEX TS-program = pdflatexmk
% !TEX spellcheck = en-US
% !BIB program = biber

\kant[1-3]



\end{abstract}

%%%%% ABSTRACT IN GERMAN
\begin{germanabstract}
  % !TEX root = ../main.tex
% !TEX encoding = UTF-8 Unicode
% !TEX TS-program = pdflatexmk
% !TEX spellcheck = en-US
% !BIB program = biber

\lipsum[1-3]

\end{germanabstract}

%%%%% MINI TABLES for each chapter
\dominitoc % include a mini table of contents
%\dominilof  % include a mini list of figures
%\dominilot  % include a mini list of tables

% This aligns the bottom of the text of each page.  It generally makes things look better.
\flushbottom

% This is where the whole-document ToC appears:
{%
  \setcounter{page}{0}%
  %
  \hypersetup{%
	  linkcolor=black%
  }%
  \tableofcontents%
}%
% This aligns the bottom of the text of each page.  It generally makes things look better.
\flushbottom

% Uncomment to generate a LIST OF THESIS PUBLICATIONS:
\begin{listofpublications}
  % !TEX root = ../main.tex
% !TEX encoding = UTF-8 Unicode
% !TEX TS-program = pdflatexmk
% !TEX spellcheck = en-US
% !BIB program = biber
 
%
% List of publications 1) mine used for this dissertation, 2) all of mine or just the rest
%
% publications for dissertation accreditation
\begin{refsection}
  \nocite{*}
  \printbibliography[%
    heading=subbibliography,%
    keyword={forthesis},%
    title={Dissertation Related Publications}]%
\end{refsection}

% publications not for dissertation accreditation
\begin{refsection}
  \nocite{*}
  \printbibliography[%
    heading=subbibliography,%
    keyword={notforthesis},%
    title=Other Publications]%
\end{refsection}


  \mtcaddchapter % \mtcaddchapter is needed when adding a non-chapter (but chapter-like) entity to avoid confusing minitoc
\end{listofpublications}
% This aligns the bottom of the text of each page.  It generally makes things look better.
\flushbottom

% Uncomment to generate a list of figures:
{%
  \hypersetup{%
	  linkcolor=black%
  }%
  \listoffigures
  \mtcaddchapter % \mtcaddchapter is needed when adding a non-chapter (but chapter-like) entity to avoid confusing minitoc
}%
% This aligns the bottom of the text of each page.  It generally makes things look better.
\flushbottom

% Uncomment to generate a list of tables:
{%
  \hypersetup{%
	  linkcolor=black%
  }%
  \listoftables%
  \mtcaddchapter
}%
% This aligns the bottom of the text of each page.  It generally makes things look better.
\flushbottom

%%%%% LIST OF ABBREVIATIONS, uncomment if you want one 
\input{text/abbreviations}
% This aligns the bottom of the text of each page.  It generally makes things look better.
\flushbottom

% end roman page numbering
\end{romanpages}

% This aligns the bottom of the text of each page.  It generally makes things look better.
\flushbottom

%%%%% CHAPTERS
% !TEX root = ../main.tex
% !TEX encoding = UTF-8 Unicode
% !TEX TS-program = pdflatexmk
% !TEX spellcheck = en-US
% !BIB program = biber


% want quotes?
\begin{savequote}[8cm]
Computers are incredibly fast, accurate and stupid; humans are incredibly slow, inaccurate, and brilliant; together they are powerful beyond imagination
%
\qauthor{--- Albert Einstein}
%
\end{savequote}

\chapter{Introduction}\label{ch:1-intro}%
%

% want an abstract per chapter??
\begin{chapterabstract}
  \kant[4]
\end{chapterabstract}

% want a toc per chapter??
\minitoc

% here goes the content
\kant[5]
Some very important \index{Concept}concept is explained here. See \autoref{fig:sample}.

\begin{figure}[htb]
  \centering
  \includegraphics[width=.6\linewidth]{neuron}
  \caption[Some shorter caption for the LOF]{Some potentially very long caption for an image or table.}\label{fig:sample}
\end{figure}

\section{A}
\kant[6-10]

\section{B}
\kant[11]

\subsection{BA}
\kant[12-14]

\subsection{BB}
\kant[14-16]
Some very important \index{Concept}concept is explained again. See \autoref{tab:sample}.

\begin{table}
  \centering
  \begin{tabular}{lll}
    \toprule
     & A & B \\
    \midrule
    C & 1 & 2 \\
    D & 3 & 4 \\
    \bottomrule
  \end{tabular}\caption[shorter caption]{potentially very long caption}\label{tab:sample}
\end{table}

\subsubsection{BBA}
\kant[16-18]

\subsubsection{BBB}
\kant[18-20]

\section{C}
\kant[20-23]

%%%% %%%%%%%%% %%%%%%%%% %%%%%%%%% %%%%%%%%%
%%%%%%%%% %%%%%%%%% %%%%%%%%% %%%%%%%%% %%%%

% !TEX root = ../main.tex
% !TEX encoding = UTF-8 Unicode
% !TEX TS-program = pdflatexmk
% !TEX spellcheck = en-US
% !BIB program = biber

\begin{savequote}[8cm]
... semantic structure of natural languages evidently offers many mysteries
  \qauthor{--- Noam Chomsky (1965)}
\end{savequote}

\chapter{Background}\label{ch:background}

\minitoc

\section{Introduction}

\kant[10] 

\section{Knowledge Engineering}

\citet{sowa-2000-knowledgerepr} proposed something very important.

In \citep{turing-1948-intelligentmachinery} we/I introduced another important thing.

We/I did other stuff too \citep{turing-1950-computingmachinery}.


Entpropy is a measure for chaos \cite{shannon-1948-entropy}.

Transfer Learning is an important concept \cite{ruder-2019-phdthesis,ruder-2019-transferlearning} which uses deep learning \citep{goodfellow-2016-dlbook}.



%% APPENDICES %%
% Starts lettered appendices, adds a heading in table of contents, and adds a
%    page that just says "Appendices" to signal the end of your main text.
\startappendices
% Add or remove any appendices you'd like here:
% !TEX root = ../main.tex
% !TEX encoding = UTF-8 Unicode
% !TEX TS-program = pdflatexmk
% !TEX spellcheck = en-US
% !BIB program = biber

\chapter{\label{app:1} Additional Material}

\minitoc

THIS IS THE APPENDIX

\par

{ 
  \includegraphics[width=.5\linewidth]{heart.png}
  \captionof{figure}{a heart}\label{fig:heart}
}




%%%%% REFERENCES
% JEM: Quote for the top of references (just like a chapter quote if you're using them).  Comment to skip.
\begin{savequote}[8cm]
The first kind of intellectual and artistic personality belongs to the hedgehogs, the second to the foxes \dots
  \qauthor{--- Sir Isaiah Berlin (2013)}
\end{savequote}
% use single-space References
\setlength{\baselineskip}{0pt}
{%
	\renewcommand*\MakeUppercase[1]{#1}%
	\nocite{*}
	\printbibliography[heading=bibintoc,title={\bibtitle}]
}%

%%%%% INDEX
\printindex

\end{document}
